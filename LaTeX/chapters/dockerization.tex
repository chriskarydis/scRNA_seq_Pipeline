\chapter{\foreignlanguage{english}{Dockerization} της Εφαρμογής}

Η εφαρμογή έχει υλοποιηθεί ώστε να τρέχει πλήρως απομονωμένα μέσω \foreignlanguage{english}{Docker}, διευκολύνοντας την εγκατάσταση και εκτέλεση σε οποιοδήποτε περιβάλλον.

\section*{Αρχείο \foreignlanguage{english}{Dockerfile}}

Το \foreignlanguage{english}{Dockerfile} βασίζεται στην επίσημη εικόνα \foreignlanguage{english}{\texttt{python:3.11-slim}} και περιλαμβάνει:

\begin{itemize}
  \item Εγκατάσταση εξαρτήσεων από το αρχείο \foreignlanguage{english}{\texttt{requirements.txt}}
  \item Αντιγραφή όλων των απαραίτητων αρχείων στον \foreignlanguage{english}{container}
  \item Ορισμός \foreignlanguage{english}{entrypoint} με την εντολή: \foreignlanguage{english}{\texttt{CMD ["streamlit", "run", "main.py"]}}
\end{itemize}

\section*{Αρχείο \foreignlanguage{english}{requirements.txt}}

Το αρχείο \foreignlanguage{english}{\texttt{requirements.txt}} περιλαμβάνει όλες τις βιβλιοθήκες που απαιτούνται για την εκτέλεση της εφαρμογής. Ενδεικτικά:

\begin{itemize}
  \item \foreignlanguage{english}{streamlit, scanpy, anndata, matplotlib, numpy, pandas}
  \item \foreignlanguage{english}{scipy, plotly, seaborn, harmonypy}
  \item \foreignlanguage{english}{python-igraph, leidenalg, openpyxl, xlsxwriter}
\end{itemize}

\newpage

\section*{Εντολές Εκτέλεσης}

Η δημιουργία και εκτέλεση του \foreignlanguage{english}{Docker container} γίνεται με:

\begin{center}
\begin{alltt}
\selectlanguage{english}
docker build -t scrna-app .

docker run -p 8501:8501 scrna-app
\selectlanguage{greek}
\end{alltt}
\end{center}

Αυτό καθιστά την εφαρμογή διαθέσιμη στη διεύθυνση: \foreignlanguage{english}{\texttt{http://localhost:8501}}

\section*{Αρχείο \foreignlanguage{english}{.dockerignore}}

Το αρχείο \foreignlanguage{english}{\texttt{.dockerignore}} περιλαμβάνει τα εξής:

\begin{itemize}
  \item \foreignlanguage{english}{\texttt{.git/}} – αποφυγή μεταφοράς \foreignlanguage{english}{git} ιστορικού
  \item \foreignlanguage{english}{\texttt{.vscode/}} – ρυθμίσεις \foreignlanguage{english}{editor}
  \item \foreignlanguage{english}{\texttt{\_\_pycache\_\_/}} – προερμηνευμένα αρχεία \foreignlanguage{english}{Python}
\end{itemize}

Με αυτόν τον τρόπο μειώνεται το μέγεθος του \foreignlanguage{english}{Docker image} και διασφαλίζεται καθαρό περιβάλλον εκτέλεσης.
